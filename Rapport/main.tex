\documentclass[12pt,twoside]{report}
\usepackage[utf8]{inputenc} % set encoding to UTF8
\usepackage[T1]{fontenc} % more info visit : https://tex.stackexchange.com/questions/664/why-should-i-use-usepackaget1fontenc
\usepackage{graphicx} % package to load images
\usepackage[french]{babel} % set language to french
\usepackage{titlesec} % customise the look of the start of the chapters
\usepackage[a4paper,width=150mm,top=25mm,bottom=25mm,bindingoffset=6mm]{geometry} % set papersize and margins
\setlength{\headheight}{14.49998pt} % needed for fancyhdr
\usepackage{fancyhdr} % add headers and footers 
\pagestyle{fancy} % header type
\renewcommand{\chaptermark}[1]{% set custom header 
\markboth{\MakeUppercase{ #1}}{}} % set default style of the header more details visit : https://mirror.marwan.ma/ctan/macros/latex/contrib/fancyhdr/fancyhdr.pdf Page 19
\fancyfoot{} % set footer
\renewcommand{\footrulewidth}{0.3pt}% % set line before footer
\fancyfoot[LE,RO]{\tiny{ENSIAS}} % LE left EVEN RO Right ODD write ENSIAS
\fancyfoot[lo,RE]{\thepage} % write the page number
\usepackage{caption} % for captioning
\usepackage{subcaption} % for multiple figs 
\usepackage{biblatex} % for references
\addbibresource{ref.bib} % set file with refs
\usepackage{csquotes} % biblatex recommend it
\usepackage{lipsum} % generate dummy text
\usepackage{wrapfig}
% for UML diagrammes
\usepackage{tikz-uml}
% set default path to images
\graphicspath{ {Images/}}  


% set title
\title{application Web/Mobile de suivi de Budget }
% set custom look to every chapter
% more details visit : https://www.overleaf.com/learn/latex/Sections_and_chapters
\titleformat
{\chapter} % command
[display] % shape
{\bfseries\Large} % format
{} % label
{0.1ex} % sep
{
  %  \rule{\textwidth}{1pt}
   % \vspace{1ex}
} % before-code
[
\vspace{-0.5ex}%
\rule{\textwidth}{0.3pt}
] % after-code

% rename tables
%\renewcommand*\contentsname{Table des matieres}
%\renewcommand{\listfigurename}{List des Figures}
%\renewcommand{\listtablename}{Liste des tables}
\setcounter{tocdepth}{4}
\setcounter{secnumdepth}{4}

\begin{document}
% title page    
\begin{titlepage}
    \begin{minipage}[t]{0.2\textwidth}    %% b or t, default is c
      \centering\includegraphics[width=\linewidth]{ensias.png}
    \end{minipage}%
  \begin{minipage}[t][2cm]{0.5\textwidth}
    \centering\bfseries\large
    \hfill
  \end{minipage}%
    \begin{minipage}[t]{0.2\textwidth}
      \centering\includegraphics[width=\linewidth]{um5.png}
  \end{minipage}
  \begin{center}
  \vspace{0.2cm}
  {\huge Ecole Nationale Supérieure d'Informatique et d'Analyse des Systèmes}\\
  \vspace{0.5cm}
  \textbf{FILIERE : GENIE LOGICIEL}
  \vspace{0.5cm}
  \hrule
  \vspace{0.2cm}
  \textbf{\huge application Web/Mobile de suivi de Budget}
  \vspace{0.5cm}
  \hrule
  \vspace{1.5cm}
  \vfill
  \begin{minipage}[t]{0.45\textwidth}    %% b or t, default is c
    \raggedright\textit{Realise par :}\\

        BAKHOUCH Nisrine\\
        EL BOUZIYANI Anas
    \end{minipage}%
    \begin{minipage}[t]{0.45\textwidth}    %% b or t, default is c
      \raggedright\textit{Sous la direction de :}\\
        
        Pr EL HAMLAOUI Mahmoud
    \end{minipage}%
       \vspace{1.8cm}
       Department Informatique\\
       ENSIAS - UM5\\
       Année Académique 2023/2024
   \end{center}
\end{titlepage}
% add a blank page
\shipout\null
\stepcounter{page} % in order to count for the blank page
%-- les differents parties --
\chapter*{Remerciments} % add or remove * to enable/disable numbering
\addcontentsline{toc}{chapter}{Remerciments}
%\textit{Au terme de ce projet nous exprimons notre profonde gratitude envers notre encadrant Pr.EL HAMLAOUI Mahmoud pour sa vigilance constante  et ses remarques constructives qui ont enrichi notre travail. Nous le remercions chaleureusement pour sa disponibilité et son précieux accompagnement, et nous avons été honorés de bénéficier de ses directives et  ses conseils éclairés tout au long du projet.}
\begin{center}
\textit{Au moment où nous pensons avoir fait œuvre utile, qu’il nous soit permis d’exprimer nos vifs remerciements et notre profonde gratitude à Monsieur le Professeur HAMLAOUI Mahmoud, qui a bien voulu et en dépit de ses multiples engagements tant professionnels que personnels, accepter de diriger et de parrainer ce travail. Qu’il veuille bien nous permettre de rendre hommage à ses qualités exceptionnelles, à son savoir-faire, et à sa contribution bénéfique à la réalisation de ce travail.}
\par
\textit{Nos remerciements vont aussi et de manière respectueuse et combien reconnaissante au grand Monsieur le Professeur NASSAR Mahmoud membre éminent de notre jury, qui nous a consacré son précieux temps afin d'assister et soutenir notre projet.}
\par
\textit{Par la même occasion, nous tenons à remercier très chaleureusement les Forces Armées Royales, la Direction de l’Enseignement Supérieur, et la totalité du personnel de l'ENSIAS pour tous les efforts louables qu’ils ont consentis pour nous permettre de réussir ce modeste travail, digne de lauréats de cette prestigieuse école.}
\end{center}


\chapter*{Resume} % add or remove * to enable/disable numbering
\addcontentsline{toc}{chapter}{Resume}
Notre projet académique de fin d'année se concentre sur le développement d'une application mobile dédiée à la gestion de syndic de copropriété afin de digitaliser les tâches manuelles et de renforcer la transparence des opérations.
Pour atteindre ces objectifs, une approche Agile, en particulier la méthodologie Scrum, a été adoptée pour gérer le projet de manière itérative.

En termes de réalisation nous avons utilisé <<UML>> pour la conception, et nous avons adopté les meilleures pratiques de développement logiciel ainsi que les technologies modernes telles que  <<Android Jetpack Compose>> pour le développement de l'interface utilisateur, <<Gradle>> pour la gestion de build et <<Firebase>> pour les services backend.

Le résultat final est une application mobile fonctionnelle offrant une interface utilisateur intuitive et conviviale.
\vfill
\hrule
\vspace{0.5cm}
\textbf{Mots clés : } Agile, Application, Android, Dagger-Hilt, Firebase, Gradle, JetPack-Compose, Mobile, Scrum, UML.
\vspace{0.5cm}
\hrule


\chapter*{Abstract} % add or remove * to enable/disable numbering
\addcontentsline{toc}{chapter}{Abstract}
The main focus of our academic project is the development of a mobile application for managing a joint property in order to optmize the existing processes of budget managing and enforce the transparency of operations. In order to achieve our goals, we opted for a Scrum/Agile approach, Meanwhile we used the latest technologies (Android JetPack Compose for ui, MVVM as a model, Firebase for backend...) the end result is a fully functional application

\vfill
\hrule
\vspace{0.5cm}
\textbf{Keywords : } Agile, Application, Android, Dagger-Hilt, Firebase, Gradle, JetPack-Compose, Mobile, Scrum, UML.
\vspace{0.5cm}
\hrule


\tableofcontents % table des matieres
\listoffigures % table des figures
\addcontentsline{toc}{chapter}{Table des figures}
\listoftables % table des tableau
\addcontentsline{toc}{chapter}{Liste des tableax}


\chapter{Introduction générale} % add or remove * to enable/disable numbering
%\addcontentsline{toc}{chapter}{Introduction générale}

\lipsum[3-27]
\cite{HENKEL2006953}
\cite{knuthwebsite}
\cite{einstein}

\chapter{Contexte générale  de projet} % add or remove * to enable/disable numbering
\section{Problematique}
\lipsum[2-2]
\section{Solution}
\lipsum[2-2]
\section{Analyse des besoins}
\lipsum[2-2]
\subsection{Besions fonctionnels}
\lipsum[2-2]
\subsection{Besoins non fonctionnels}
\lipsum[2-2]
\section{Objectifs}
\lipsum[2-2]

\chapter{Conception de Projet}
\section{Conception globale}
Dans le cadre de notre projet de developpement  d'une application de gestion de syndic de copropriété, la phase de conception de l'app revêt une importance particulière et cruciale pour garantir sa fonctionnalité et son adaptation aux besoins des utilisateurs. 
Cette section du rapport se concentre sur la manière dont nous avons planifié, conçu et structuré l'application pour répondre aux aux normes de l'industrie en termes de conception et de développement logiciel et aux attentes des utilisateurs
\subsection{méthodologie SCRUM/Agile}
\subsubsection{Agile manifesto}

%\textbf{Manifesto for Agile Software Development }\\
%We are uncovering better ways of developing
%software by doing it and helping others do it.
%Through this work we have come to value:
"
\begin{enumerate}
    \item  Individuals and interactions over processes and tools
    \item Working software over comprehensive documentation
    \item Customer collaboration over contract negotiation
    \item Responding to change over following a plan 
\end{enumerate}
"
\cite*{beck2001agile}
%That is, while there is value in the items on the right, we value the items on the left more. 

\subsubsection{Scrum}
\begin{wrapfigure}{r}{0.25\textwidth} 
    \centering
    \includegraphics[width=0.25\textwidth]{scrum logo.png}
    \caption{logo de scrum Agile}
\end{wrapfigure}
Selon The Scrum GuideTM, Scrum est « un framework léger qui aide les personnes, les équipes et les organisations à générer de la valeur grâce à des solutions adaptatives à des problèmes complexes.1 » Scrum est le framework agile le plus largement utilisé et le plus populaire. Le terme agile décrit un ensemble spécifique de principes et de valeurs fondamentaux pour l'organisation et la gestion d'un travail complexe.
Bien qu'il ait ses racines dans le développement de logiciels, Scrum fait aujourd'hui référence à un cadre léger utilisé dans tous les secteurs pour fournir des produits et services complexes et innovants qui ravissent réellement les clients. C'est simple à comprendre, mais difficile à maîtriser.
\cite*{scrumalliance}
\subsection{Languages et technologies Utilisées}
TO BE IMPLEMENTED
\subsubsection{Android}
\begin{wrapfigure}{r}{0.25\textwidth} 
    \centering
    \includegraphics[width=0.25\textwidth]{android-logo.png}
    \caption{logo de scrum Agile}
\end{wrapfigure}
Android est un système d'exploitation mobile open source fondé sur le noyau Linux et développé par un consortium d'entreprises, le Open Handset Alliance, sponsorisé par Google. 
Android est défini comme étant une pile de logiciels, c'est-à-dire un ensemble de logiciels destinés à fournir une solution clé en main pour les appareils mobiles, smartphones et tablettes tactiles. Cette pile comporte un système d'exploitation (comprenant un noyau Linux), les applications clés telles que le navigateur web, le téléphone et le carnet d'adresses ainsi que des logiciels intermédiaires entre le système d'exploitation et les applications
\cite*{wiki:Android}
\subsubsection{les bonnes pratiques}
TO BE IMPLEMENTED
\subsubsection{Kotlin}
\begin{wrapfigure}{r}{0.25\textwidth} 
    \centering
    \includegraphics[width=0.25\textwidth]{kotlin.png}
    \caption{logo de scrum Agile}
\end{wrapfigure}
Kotlin est un langage de programmation orienté objet et fonctionnel, avec un typage statique qui permet de compiler pour la machine virtuelle Java, JavaScript, et vers plusieurs plateformes en natif (grâce à LLVM). Son développement provient principalement d'une équipe de programmeurs chez JetBrains basée à Saint-Pétersbourg en Russie (son nom vient de l'île de Kotline, près de St. Pétersbourg).
Google annonce pendant la conférence Google I/O 2017 que Kotlin devient le second langage de programmation officiellement pris en charge par Android après Java. Le 8 mai 2019, toujours lors de la conférence Google I/O, Kotlin devient officiellement le langage de programmation voulu et recommandé par le géant américain Google pour le développement des applications Android.
Pivotal Software annonce le 4 janvier 2017 le support officiel de Kotlin sur la cinquième version du Framework Spring. 
\cite*{wiki:Kotlin}
\subsubsection{Android JetPack Compose}
TO BE IMPLEMENTED
\subsubsection{daggerHilt (injecteur de dependances)}
TO BE IMPLEMENTED
\subsubsection{Gradle}
TO BE IMPLEMENTED
\subsubsection{Firebase}
TO BE IMPLEMENTED
\subsubsection{Firebase FireAuth}
TO BE IMPLEMENTED
\subsubsection{Firebase FireStore}
TO BE IMPLEMENTED


\subsection{les diagrammes UML}
La modélisation  joue un rôle crucial  dans le développement logiciel,permettant de représenter de manière claire et concise les différents aspects d'un système informatique. 
Parmi les outils de modélisation les plus répandus, on trouve les diagrammes de cas d'utilisation et de classes
\newpage
\begin{figure}[h!]
        \begin{tikzpicture}
        
        \umlclass[x=0,y=0,type=interface,scale = 0.55,fill=blue!10]{AccountService}{
          }{ 
            + authenticate (login : LoginUiModel, onResult: (User) -> Unit) :Unit\\
            + logout() : Unit \\
            + Register(Register: RegisterUiModel, onResult: (User) -> Unit) : Unit \\
            + reset(email: String,onResult: () -> Unit) : Unit \\
          }
          \umlclass[x=9,y=0,scale = 0.55,fill=blue!10]{FireBaseAccountService}{
          }{ 
            - resetPasswordListener(task: Task<Void>, onResult: () -> Unit) :Unit\\
            - loginListerner(task: Task<AuthResult>,onResult: (User) -> Unit) : Unit \\
            - registerListerner(t:Task<AuthResult>,r:RegisterUiModel,onResult:(User)->Unit) :Unit \\
            - setUserData(t:Task<AuthResult>,u:String?,r:RegisterUiModel,onResult:(User)->Unit) : Unit \\
            - authException(e:Exception) :AuthException\\
            - getUserData(task:Task<AuthResult>,uid:String?,onResult:(User) -> Unit) :Unit\\
            - ongetUserDataSucessListener(document: DocumentSnapshot,onResult: (User) -> Unit) :Unit\\
            - onFirestoreException(e: java.lang.Exception) :Unit\\
          }
          \umlclass[x=1,y=-3,scale = 0.55,,fill=green!10]{RegisterUiModel}{
            + prenom : String \\
            + nom : String \\
            + email : String \\
            + password : String
          }{}
            \umlclass[x=-2,y=-3,scale = 0.55,fill=green!10]{LoginUiModel}{
            + email : String \\
            + password : String
          }{}
              \umlclass[x=1,y=-6,scale = 0.55,fill=orange!10]{LoginUiState}{
            + email : String \\
            + password : String \\
            + logging : Boolean \\
            + validMail : Boolean
          }{}
              \umlclass[x=4,y=-6,scale = 0.55,fill=orange!10]{ResetUiState}{
            + email : String \\
            + isMailValid : Boolean
          }{}
          
              \umlclass[x=-2,y=-6,scale = 0.55,fill=orange!10]{RegisterUiState}{
            + prenom : String \\
            + nom : String \\
            + email : String \\
            + password : String \\
            + validMail : Boolean
          }{}
        
          \umlclass[x=8,y=-3,scale = 0.55,fill=red!10]{Exception}{
          }{
          } 
          
          \umlclass[x=8,y=-5,type=abstract,scale = 0.55,fill=red!10]{AuthException}{
          }{
            + getmessage() : Int
          }
            \umlclass[x=12,y=-3,scale = 0.55,fill=red!10]{DeadLineExceeded}{
          }{
          }
             \umlclass[x=12,y=-4,scale = 0.55,fill=red!10]{InvalidCredentialsException}{
          }{
          }
        
           \umlclass[x=12,y=-5,scale = 0.55,fill=red!10]{InvalidUserIdException}{
          }{
          }
        
          \umlclass[x=12,y=-6,scale = 0.55,fill=red!10]{MalFormatedEmailException}{
          }{
          }
        
            \umlclass[x=12,y=-7,scale = 0.55,fill=red!10]{RegisterPasswordMismatchException}{
          }{
          }
        
              \umlclass[x=12,y=-8,scale = 0.55,fill=red!10]{UndefinedException}{
          }{
          } 
        
          \umlclass[x=12,y=-9,scale = 0.55,fill=red!10]{UserDataMissingException}{
          }{
          } 
        
            \umlclass[x=4,y=-3,scale = 0.55,fill=white!10]{User}{
            - IS\_ADMIN:Boolean \\
            name : String \\
            familyname : String \\
            id : String \\
            email : String
          }{
          } 
        \umlclass[x=8,y=-7,scale = 0.55,fill=yellow!10]{ViewModel}{
          }{
          } 
        
        \umlclass[x=4,y=-11,scale = 0.55,fill=yellow!10]{AuthViewModel}{
            + loginUistate :  MutableState<LoginUiState> \\
            + registerUistate : MutableState<RegisterUiState> \\
            + resetUiState : MutableState<ResetUiState> \\
          }{
            +Constructor(accountService : AccountService) \\ % TODO: find the correct way to represent the constructor
            + login(openAndPopUp:(String,String)->Unit,toggleAdminUservalues:(isadmin:Boolean,logged:Boolean)->Unit)  : Unit \\
            + loginExceptionHandler(e: AuthException) : Unit \\
            + setUser(user: User) : Unit \\
            + signupscreen(open: ( String) -> Unit) : Unit \\
            + setLoginEmail(newemail:String) : Unit \\
            + setLoginPassword(newpass:String) : Unit \\
            + onLoginEmailValidation(valid: Boolean) :  Unit \\
            + register(openAndPopUp: (String, String) -> Unit) : Unit \\
            + setRegisterName(s: String) : Unit \\
            + setFamilyname(s: String) : Unit \\
            + setEmail(s: String) : Unit \\
            + setRegisterPass(s: String) : Unit \\
            + setVerificationPass(s: String) : Unit \\
            + onRegisterEmailValidation(valid: Boolean) : Unit \\
            + resetSetEmail(s: String) : Unit \\
            + resetPassword(openAndPopUp: (String, String) -> Unit) : Unit \\
            + onResetEmailValidation(b: Boolean) : Unit \\
            + resetPasswordScreen(open: (String) -> Unit) : Unit 
          } 
        
        \umlinherit{AuthViewModel}{ViewModel}
        \umlinherit{FireBaseAccountService}{AccountService}
        \umlinherit{AuthException}{Exception}
        \umlinherit{DeadLineExceeded}{AuthException}
        \umlinherit{InvalidCredentialsException}{AuthException}
        \umlinherit{InvalidUserIdException}{AuthException}
        \umlinherit{MalFormatedEmailException}{AuthException}
        \umlinherit{RegisterPasswordMismatchException}{AuthException}
        \umlinherit{UndefinedException}{AuthException}
        \umlinherit{UserDataMissingException}{AuthException}
        
        \umlaggreg[]{AuthViewModel}{LoginUiState}
        \umlaggreg[]{AuthViewModel}{RegisterUiState}
        \umlaggreg[]{AuthViewModel}{ResetUiState}
        
        \umlassoc[]{User}{AccountService}
        \umlassoc[]{AuthException}{AuthViewModel}
        \umlassoc[]{User}{AuthViewModel}
        \umlassoc[]{LoginUiModel}{AuthViewModel}
        \umlassoc[]{RegisterUiModel}{AuthViewModel}
        \umlassoc [geometry=|-, anchors=-160 and 170,] {AccountService}{AuthViewModel}
        
        %\umlunicompo[geometry=-|, arg=titi, mult=*, pos=1.7, stereo=vector]{D}{C}
        %\umlimport[geometry=|-, anchors=90 and 50, name=import]{sp2}{sp1}
        %\umlinherit[geometry=-|]{D}{B}
        \end{tikzpicture}
        \caption{le diagramme de Class pour Sprint 1}
        \label{fig : Class diagram Sprint 1}
    \end{figure}

\subsubsection{Diagramme de cas d'utilisdation}
Le  Diagramme de cas d'utilisation est une représentation graphique des interactions entre les utilisateurs -acteurs- et le système logiciel
\input{Diagrammes/generalUsecase.tex}
\begin{figure}[h]
    \centering
    \begin{tikzpicture}
        \begin{umlsystem}[x=3, fill=blue!10]{gestion des cotisations }
            \umlusecase[name = addContribution]{ajouter cotisation}
            \umlusecase[name = modifycotribution, y = -2]{rectifier cotisation} 
            \umlusecase[name = deletecontribution, y = -4]{supprimer cotisation}
            \umlusecase[name = auth,x = 4,y = -1]{s'authentifier}  
            \end{umlsystem}
        
            \umlactor[x =-2,y=-3]{administrateur}
        
        \umlassoc{administrateur}{addContribution}
        \umlassoc{administrateur}{modifycotribution}
        \umlassoc{administrateur}{deletecontribution}

        \umlHVinclude[name=incl,anchor1=0 , anchor2=40]{addContribution}{auth}
        \umlHVinclude[name=incl]{modifycotribution}{auth}
        \umlHVinclude[name=incl,anchor1=0 , anchor2=350]{deletecontribution}{auth}
        %\umlHVinclude[name=incl,anchor1=30 , anchor2=300]{addBudget}{auth}
 
    \end{tikzpicture}
    \caption{le diagramme de cas d'utilisation detaillé pour les cotisations}
    \label{fig : usecase 3}
\end{figure}

%and this is a reference to the usecase fig : \ref{fig : usecase 3}
\begin{figure}[h]
    \centering
    \begin{tikzpicture}
        \begin{umlsystem}[x=3, fill=green!10]{gestion des dépenses }
            \umlusecase[name = addSpending]{ajouter dépense}
            \umlusecase[name = addSpendingtype, y = -2]{ajouter type de dépense}
            \umlusecase[name = modifySpending, y = -4]{rectifier dépense} 
            \umlusecase[name = deleteSpending, y = -6]{supprimer dépense}
            \umlusecase[name = auth,x = 5,y = -1]{s'authentifier}  
            \end{umlsystem}
        
            \umlactor[x =-2,y=-3]{administrateur}
        
        \umlassoc{administrateur}{addSpending}
        \umlassoc{administrateur}{addSpendingtype}
        \umlassoc{administrateur}{modifySpending}
        \umlassoc{administrateur}{deleteSpending}
        \umlHVinclude[name=incl,anchor1=0 , anchor2=40]{addSpending}{auth}
        \umlHVinclude[name=incl]{addSpendingtype}{auth}
        \umlHVinclude[name=incl,anchor1=0 , anchor2=350]{modifySpending}{auth}
        \umlHVinclude[name=incl,anchor1=0 , anchor2=320]{deleteSpending}{auth}
        %\umlHVinclude[name=incl,anchor1=30 , anchor2=300]{addBudget}{auth}
 
    \end{tikzpicture}
    \caption{le diagramme de cas d'utilisation detaillé pour les dépenses}
    \label{fig : usecase 2}
\end{figure}

%and this is a reference to the usecase fig : \ref{fig : usecase 2}
\subsubsection{Diagramme de séquences}
TO BE IMPLEMENTED
\subsubsection{Diagramme de classe}
Le diagramme de classe est le plan du système .utiliser pour modéliser les objets qui constituent le système, pour afficher les relations entre les objets et pour décrire ce que ces objets font et les services qu'ils fournissent.
\cite{IBM:class}

\chapter{Realisation}
\section{Sprint 0: Mise en Oeuvre}
\subsection{identification des acteurs}
\subsection{userStories}
\subsection{prototypage des interfaces}

\section{Sprint 1}
\subsection{specification fonctionnel}
\subsection{Sprint Backlog}
\subsection{conception}
\subsubsection{diagramme de cas d'utilisation}
\begin{figure}[h!]
    \begin{center}
    \begin{tikzpicture}
        \begin{umlsystem}[x=3, fill=red!10]{Sprint1}
            \umlusecase[name = CONNECT]{se connecter}
            \umlusecase[name = SIGNUP, y = -2]{s'inscrire}
            \umlusecase[name = renewPassword, y = -4]{renouvler le mot de passe} 
            \umlusecase[name = AUTH,x = 5,y = -1]{s'authentifier}  
            \end{umlsystem}
        
            \umlactor[x=-2]{utilisateur}
            \umlactor[x =-2,y=-3]{administrateur}
        
        \umlassoc{utilisateur}{CONNECT}
        \umlassoc{utilisateur}{SIGNUP}
        \umlassoc{utilisateur}{renewPassword}
        \umlassoc{administrateur}{CONNECT}
        \umlassoc{administrateur}{renewPassword}

        \umlHVinclude[name=incl]{CONNECT}{AUTH}
       % \umlHVinclude[name=incl]{addSpending}{auth}
        %\umlHVinclude[name=incl,anchor1=30 , anchor2=300]{addBudget}{auth}

    \end{tikzpicture}
    \caption{le diagramme de cas d'utilisation pour Sprint 1}
\end{center}
    \label{fig : Sprint 1 usecase }
\end{figure}

%and this is a reference to the usecase fig : \ref{fig : usecase 1}
\subsubsection{diagramme de classe}
\begin{figure}[h!]
        \begin{tikzpicture}
        
        \umlclass[x=0,y=0,type=interface,scale = 0.55,fill=blue!10]{AccountService}{
          }{ 
            + authenticate (login : LoginUiModel, onResult: (User) -> Unit) :Unit\\
            + logout() : Unit \\
            + Register(Register: RegisterUiModel, onResult: (User) -> Unit) : Unit \\
            + reset(email: String,onResult: () -> Unit) : Unit \\
          }
          \umlclass[x=9,y=0,scale = 0.55,fill=blue!10]{FireBaseAccountService}{
          }{ 
            - resetPasswordListener(task: Task<Void>, onResult: () -> Unit) :Unit\\
            - loginListerner(task: Task<AuthResult>,onResult: (User) -> Unit) : Unit \\
            - registerListerner(t:Task<AuthResult>,r:RegisterUiModel,onResult:(User)->Unit) :Unit \\
            - setUserData(t:Task<AuthResult>,u:String?,r:RegisterUiModel,onResult:(User)->Unit) : Unit \\
            - authException(e:Exception) :AuthException\\
            - getUserData(task:Task<AuthResult>,uid:String?,onResult:(User) -> Unit) :Unit\\
            - ongetUserDataSucessListener(document: DocumentSnapshot,onResult: (User) -> Unit) :Unit\\
            - onFirestoreException(e: java.lang.Exception) :Unit\\
          }
          \umlclass[x=1,y=-3,scale = 0.55,,fill=green!10]{RegisterUiModel}{
            + prenom : String \\
            + nom : String \\
            + email : String \\
            + password : String
          }{}
            \umlclass[x=-2,y=-3,scale = 0.55,fill=green!10]{LoginUiModel}{
            + email : String \\
            + password : String
          }{}
              \umlclass[x=1,y=-6,scale = 0.55,fill=orange!10]{LoginUiState}{
            + email : String \\
            + password : String \\
            + logging : Boolean \\
            + validMail : Boolean
          }{}
              \umlclass[x=4,y=-6,scale = 0.55,fill=orange!10]{ResetUiState}{
            + email : String \\
            + isMailValid : Boolean
          }{}
          
              \umlclass[x=-2,y=-6,scale = 0.55,fill=orange!10]{RegisterUiState}{
            + prenom : String \\
            + nom : String \\
            + email : String \\
            + password : String \\
            + validMail : Boolean
          }{}
        
          \umlclass[x=8,y=-3,scale = 0.55,fill=red!10]{Exception}{
          }{
          } 
          
          \umlclass[x=8,y=-5,type=abstract,scale = 0.55,fill=red!10]{AuthException}{
          }{
            + getmessage() : Int
          }
            \umlclass[x=12,y=-3,scale = 0.55,fill=red!10]{DeadLineExceeded}{
          }{
          }
             \umlclass[x=12,y=-4,scale = 0.55,fill=red!10]{InvalidCredentialsException}{
          }{
          }
        
           \umlclass[x=12,y=-5,scale = 0.55,fill=red!10]{InvalidUserIdException}{
          }{
          }
        
          \umlclass[x=12,y=-6,scale = 0.55,fill=red!10]{MalFormatedEmailException}{
          }{
          }
        
            \umlclass[x=12,y=-7,scale = 0.55,fill=red!10]{RegisterPasswordMismatchException}{
          }{
          }
        
              \umlclass[x=12,y=-8,scale = 0.55,fill=red!10]{UndefinedException}{
          }{
          } 
        
          \umlclass[x=12,y=-9,scale = 0.55,fill=red!10]{UserDataMissingException}{
          }{
          } 
        
            \umlclass[x=4,y=-3,scale = 0.55,fill=white!10]{User}{
            - IS\_ADMIN:Boolean \\
            name : String \\
            familyname : String \\
            id : String \\
            email : String
          }{
          } 
        \umlclass[x=8,y=-7,scale = 0.55,fill=yellow!10]{ViewModel}{
          }{
          } 
        
        \umlclass[x=4,y=-11,scale = 0.55,fill=yellow!10]{AuthViewModel}{
            + loginUistate :  MutableState<LoginUiState> \\
            + registerUistate : MutableState<RegisterUiState> \\
            + resetUiState : MutableState<ResetUiState> \\
          }{
            +Constructor(accountService : AccountService) \\ % TODO: find the correct way to represent the constructor
            + login(openAndPopUp:(String,String)->Unit,toggleAdminUservalues:(isadmin:Boolean,logged:Boolean)->Unit)  : Unit \\
            + loginExceptionHandler(e: AuthException) : Unit \\
            + setUser(user: User) : Unit \\
            + signupscreen(open: ( String) -> Unit) : Unit \\
            + setLoginEmail(newemail:String) : Unit \\
            + setLoginPassword(newpass:String) : Unit \\
            + onLoginEmailValidation(valid: Boolean) :  Unit \\
            + register(openAndPopUp: (String, String) -> Unit) : Unit \\
            + setRegisterName(s: String) : Unit \\
            + setFamilyname(s: String) : Unit \\
            + setEmail(s: String) : Unit \\
            + setRegisterPass(s: String) : Unit \\
            + setVerificationPass(s: String) : Unit \\
            + onRegisterEmailValidation(valid: Boolean) : Unit \\
            + resetSetEmail(s: String) : Unit \\
            + resetPassword(openAndPopUp: (String, String) -> Unit) : Unit \\
            + onResetEmailValidation(b: Boolean) : Unit \\
            + resetPasswordScreen(open: (String) -> Unit) : Unit 
          } 
        
        \umlinherit{AuthViewModel}{ViewModel}
        \umlinherit{FireBaseAccountService}{AccountService}
        \umlinherit{AuthException}{Exception}
        \umlinherit{DeadLineExceeded}{AuthException}
        \umlinherit{InvalidCredentialsException}{AuthException}
        \umlinherit{InvalidUserIdException}{AuthException}
        \umlinherit{MalFormatedEmailException}{AuthException}
        \umlinherit{RegisterPasswordMismatchException}{AuthException}
        \umlinherit{UndefinedException}{AuthException}
        \umlinherit{UserDataMissingException}{AuthException}
        
        \umlaggreg[]{AuthViewModel}{LoginUiState}
        \umlaggreg[]{AuthViewModel}{RegisterUiState}
        \umlaggreg[]{AuthViewModel}{ResetUiState}
        
        \umlassoc[]{User}{AccountService}
        \umlassoc[]{AuthException}{AuthViewModel}
        \umlassoc[]{User}{AuthViewModel}
        \umlassoc[]{LoginUiModel}{AuthViewModel}
        \umlassoc[]{RegisterUiModel}{AuthViewModel}
        \umlassoc [geometry=|-, anchors=-160 and 170,] {AccountService}{AuthViewModel}
        
        %\umlunicompo[geometry=-|, arg=titi, mult=*, pos=1.7, stereo=vector]{D}{C}
        %\umlimport[geometry=|-, anchors=90 and 50, name=import]{sp2}{sp1}
        %\umlinherit[geometry=-|]{D}{B}
        \end{tikzpicture}
        \caption{le diagramme de Class pour Sprint 1}
        \label{fig : Class diagram Sprint 1}
    \end{figure}
\subsection{Realisation}
\subsubsection{interface de connexion}
\subsubsection{interface d'inscription}
\subsubsection{interface de renitialisation de mot de passe}



\section{Sprint 2}
\subsection{specification fonctionnel}
\subsection{Sprint Backlog}
\subsection{conception}
\subsubsection{diagramme de cas d'utilisation}
\begin{figure}[h]
    \centering
    \begin{tikzpicture}
        \begin{umlsystem}[x=3, fill=orange!10]{Sprint 2}
            \umlusecase[name = MONTH]{voir la situation des mois}
            \umlusecase[name = OPERATION, y = -2]{voir les opérations}
            \end{umlsystem}
            \umlactor[x=-3,y=1]{utilisateur}
            \umlactor[x =-3,y=-2]{administrateur}

            \umlinherit[]{administrateur}{utilisateur}
        \umlassoc{utilisateur}{MONTH}
        \umlassoc{utilisateur}{OPERATION}

       % \umlHVinclude[name=incl]{addSpending}{auth}
        %\umlHVinclude[name=incl,anchor1=30 , anchor2=300]{addBudget}{auth}

    \end{tikzpicture}
    \caption{le diagramme de cas d'utilisation pour Sprint 2}
    \label{fig : Sprint 2 usecase }
\end{figure}

%and this is a reference to the usecase fig : \ref{fig : usecase 1}
\subsubsection{diagramme de classe}
\begin{figure}[h]
        \begin{tikzpicture}

        \umlclass[x=-1,y=3,type=interface,scale = 0.8,fill=blue!10]{DataService}{
            + users : Flow<List<User>$ $>\\
            + expensesTypes: Flow<List<SpendType>$ $>\\
            + monthList: Flow<List<Month>$ $>\\
          }{ 
            + getOperationsFlow(id: String): Flow<List<Operation>$ $> :Unit\\ 
          }

          \umlclass[x=0,y=-2,scale=0.8,fill=blue!10]{FireBaseDataService}{
            + auth: FirebaseAuth\\
            + store : FirebaseFirestore\\
            - MONTH\_DATA\_COLLECTION : String\\
            - SPEND\_TYPES\_COLLECTION : String\\
            - LIST : String\\
          }{
            + Constructor(auth: FirebaseAuth,store : FirebaseFirestore)\\
            - getexpenseType(id:String):SpendType\\
            - getUser(id:String):User\\
            - onFirestoreException(e: java.lang.Exception) : Unit
          } 
          \umlclass[x=7,y=3,scale = 0.8,fill=white!10]{Month}{
            + id :String \\
            - prevBalance : Long \\
            + currBalance: Long \\
            + monthDate : Date \\
            + debit : Long \\
            + credit : Long
          }{
          } 

          \umlclass[x=9,y=-1,scale = 0.8,fill=white!10]{SpendType}{
            + id :String \\
            - prevBalance : Long \\
            + currBalance: Long \\
            + monthDate : Date \\
            + debit : Long \\
            + credit : Long
          }{
          } 
          \umlclass[x=8,y=-5,scale = 0.8,fill=white!10]{User}{
            - IS\_ADMIN:Boolean \\
            name : String \\
            familyname : String \\
            id : String \\
            email : String
          }{
          } 

          \umlclass[x=9,y=-9,scale = 0.8,fill=white!10]{Operation}{
            + id :String \\
            - ref:String \\
            + type :String \\
            + value : Long \\
            + date : Date  \\
            + Spendtype : SpendType \\
            + user : User \\
          }{
          } 
          \umlclass[x=0,y=-5,scale = 0.8,fill=yellow!10]{ViewModel}{
            }{
            } 
          
          \umlclass[x=1,y=-9,scale = 0.8,fill=yellow!10]{MonthViewModel}{
              - dataService: DataService\\
              + monthList :  Flow<List<Month>$ $> \\
            }{
              + constructor(dataService: DataService) \\ % TODO: find the correct way to represent the constructor
              + onMonthSelect(mId: String,m:Int,y:Int, open: (String) -> Unit): Unit \\
              + getOperationFlow(id:String?): Flow<List<Operation>$ $> \\
            } 
        
        \umlinherit[geometry=|-|,anchors=50 and -50]{MonthViewModel}{ViewModel}
        \umlinherit[geometry=|-|]{FireBaseDataService}{DataService}
        
        \umlaggreg[geometry=|-,anchors=80 and 0]{Operation}{User}
        \umlaggreg[geometry=|-,anchors=50 and 0]{Operation}{SpendType}
        \umlaggreg[geometry=-|,anchors=180 and -170]{MonthViewModel}{DataService}
        \umlaggreg[]{MonthViewModel}{Operation}
        \umlaggreg[geometry=|-,anchors=20 and 0]{MonthViewModel}{Month}
        \end{tikzpicture}
        \caption{le diagramme de Class pour Sprint 2}
        \label{fig : Class diagram Sprint 1}
    \end{figure}
\subsection{Realisation}
\subsubsection{interface de situation des mois}
\subsubsection{interface des opérations}



\section{Sprint 3}
\subsection{specification fonctionnel}
\subsection{Sprint Backlog}
\subsection{conception}
\subsubsection{diagramme de cas d'utilisation}
\begin{figure}[h]
    \centering
    \begin{tikzpicture}
        \begin{umlsystem}[x=3, fill=orange!10]{Sprint 2}
            \umlusecase[name = MONTH]{voir la situation des mois}
            \umlusecase[name = OPERATION, y = -2]{voir les opérations}
            \end{umlsystem}
            \umlactor[x=-3,y=1]{utilisateur}
            \umlactor[x =-3,y=-2]{administrateur}

            \umlinherit[]{administrateur}{utilisateur}
        \umlassoc{utilisateur}{MONTH}
        \umlassoc{utilisateur}{OPERATION}

       % \umlHVinclude[name=incl]{addSpending}{auth}
        %\umlHVinclude[name=incl,anchor1=30 , anchor2=300]{addBudget}{auth}

    \end{tikzpicture}
    \caption{le diagramme de cas d'utilisation pour Sprint 2}
    \label{fig : Sprint 2 usecase }
\end{figure}

%and this is a reference to the usecase fig : \ref{fig : usecase 1}
\subsubsection{diagramme de classe}
\begin{figure}[h!]
        \begin{tikzpicture}

        \umlclass[x=0,y=-3,type=interface,scale = 0.8,fill=blue!10]{DataService}{
            + users : Flow<List<User>$ $>\\
            + expensesTypes: Flow<List<SpendType>$ $>\\
            + monthList: Flow<List<Month>$ $>\\
          }{ 
            + getOperationsFlow(id: String): Flow<List<Operation>$ $> :Unit\\ 
            + addExpenseType(name: String, onResult: () -> Unit) : Unit\\
            + updateExpenseType(id: String, newname: String, onResult: () -> Unit) : Unit \\
            + addOperation (op :Operation, onResult: () -> Unit) : Unit \\
            + removeOperation(op: Operation, onResult: ()->Unit) : Unit 
          }

          \umlclass[x=0,y=3,scale=0.8,fill=blue!10]{FireBaseDataService}{
            + auth: FirebaseAuth\\
            + store : FirebaseFirestore\\
           % - OPVALUE : String \\
           % - OPTYPE : String \\
           % - OPDATE : String \\
           % - OPREF : String \\
           % - PREV\_BALANCE : String \\
           % - CURR\_BALANCE : String \\
           % - DEBIT : String \\
           % - CREDIT : String \\
           % - MONTHDATE : String \\
           % - LIST : String \\
           % - USERS : String \\
           % - USERS : String \\
            - MONTH\_DATA\_COLLECTION : String\\
            - SPEND\_TYPES\_COLLECTION : String\\
            - LIST : String\\
          }{
            + Constructor(auth: FirebaseAuth,store : FirebaseFirestore)\\
            - getexpenseType(id:String) : SpendType\\
            - getUser(id:String) : User\\
            - onFirestoreException(e: java.lang.Exception) : Unit\\
            - addExpenseType(name: String, onResult: () -> Unit) : Unit\\
            - updateExpenseType(id: String, newname: String, onResult: () -> Unit) : Unit\\
            - updateMonth(m: Month, onResult: () -> Unit) : Unit\\
            - addMonth(m: Month) : Month\\
            - getMonthDateBasedOnOpDate(date:Date) : Date\\
            - getMonthByDateOrCreateNewOne(time: Date): Month  
          } 
          \umlclass[x=8,y=3,scale = 0.6,fill=white!10]{Month}{
            + id :String \\
            + prevBalance : Long \\
            + currBalance: Long \\
            + monthDate : Date \\
            + debit : Long \\
            + credit : Long
          }{
          }

          \umlclass[x=9,y=0,scale = 0.6,fill=white!10]{SpendType}{
            + id : String\\
            + name : String
          }{
          } 
          \umlclass[x=8,y=-3,scale = 0.6,fill=white!10]{User}{
            - IS\_ADMIN:Boolean \\
            name : String \\
            familyname : String \\
            id : String \\
            email : String
          }{
          } 

          \umlclass[x=1,y=-7,scale = 0.6,fill=orange!10]{ContributionUiState}{
            + user : User \\
            + date : Date \\
            + amount : Int \\
            + pendingOperation : Boolean \\
          }{}

          \umlclass[x=5,y=-7,scale = 0.6,fill=orange!10]{ExpenseuiState}{
            + type : String \\
            + date : Date \\
            + amount : Int \\
            + visibleName : String \\
            + ref : String \\
            + pendingOperation : Boolean \\
          }{}

          \umlclass[x=9,y=-12,scale = 0.6,fill=white!10]{Operation}{
            + id :String \\
            - ref:String \\
            + type :String \\
            + value : Long \\
            + date : Date  \\
            + Spendtype : SpendType \\
            + user : User \\
          }{
          } 
          \umlclass[x=-3,y=-7,scale = 0.6,fill=yellow!10]{ViewModel}{
            }{
            } 
          
          \umlclass[x=-1,y=-12,scale = 0.6,fill=yellow!10]{OperationViewModel}{
              - dataService: DataService\\
              - CONTRIBUTION : String \\
              - EXPENSE : String \\
              - users : Flow<List<User>$ $> \\
              - contribiutionUiState :  MutableState<ContributionUiState> \\
              - expenseUiState : MutableState<ExpenseuiState> \\
              - expensesTypes : Flow<List<SpendType>$ $> \\
              + monthList :  Flow<List<Month>$ $> \\
            }{
              + constructor(dataService: DataService) \\ % TODO: find the correct way to represent the constructor
              + addExpense() : Unit \\
              + addexpenseResult() : Unit \\
              + addcontributionResult() : Unit\\
              + addExpenseType(id: String) : Unit \\
              + modifyExpenseType(id: String, name: String) : Unit\\
              + setNewVal(newVal: String) : Unit\\
              + onContribValueChange(newval: String) : Unit \\
              + addContribution() : Unit \\
            } 
        
        \umlinherit[geometry=|-|,anchors=50 and -50]{OperationViewModel}{ViewModel}
        \umlinherit[geometry=|-|]{FireBaseDataService}{DataService}
        
        \umlaggreg[geometry=|-,anchors=80 and 0]{Operation}{User}
        \umlaggreg[geometry=|-,anchors=50 and 0]{Operation}{SpendType}
        \umlaggreg[geometry=-|,anchors=30 and -60]{OperationViewModel}{SpendType}
        \umlaggreg[draw=black!30]{OperationViewModel}{Month}
        \umlaggreg[geometry=-|,anchors=180 and -170]{OperationViewModel}{DataService}
        \umlassoc[geometry=-|,anchors=20 and 100]{FireBaseDataService}{Month}
        \umlaggreg[geometry=-|,anchors=-20 and 120]{DataService}{Operation}
        \umlaggreg{OperationViewModel}{ContributionUiState}
        \umlaggreg{OperationViewModel}{ExpenseuiState}
        \umlassoc[geometry=-|,anchors=-20 and 130]{FireBaseDataService}{User}
        %\umlaggreg[geometry=|-,anchors=20 and 0]{MonthViewModel}{Month}
        \end{tikzpicture}
        \caption{Le diagramme de Class pour Sprint 3}
        \label{fig : Class diagram Sprint 3}
    \end{figure}

\subsection{Realisation}
\subsubsection{interface pour gérer les opération}
\paragraph{ajouter type de dépense}
\paragraph{modifier type de dépense}
\paragraph{ajouter de dépense}
\paragraph{ajouter cotisation}
\paragraph{supprimer operation}




\appendix

\addcontentsline{toc}{chapter}{ Bibliographie}
\printbibliography
\chapter{annexes} % add or remove * to enable/disable numbering

\begin{figure}[h]
    \centering
    \includegraphics[width=1\textwidth]{diag classe metier.png}
    \caption{Diagramme de classe métier}
\end{figure}


list des products backlog

\begin{figure}[h]
    \centering
    \includegraphics[width=1\textwidth]{sprint1.png}
    \label{Sprint1}
    \caption{Product Backlog de sprint 1}
\end{figure}


\begin{figure}[h]
    \centering
    \includegraphics[width=1\textwidth]{sprint2.png}
    \label{Sprint2}
    \caption{Product Backlog de sprint 2}
\end{figure}


\begin{figure}[h]
    \centering
    \includegraphics[width=1\textwidth]{sprint3.png}
    \label{Sprint3}
    \caption{Product Backlog de sprint 3}
\end{figure}






\end{document}
