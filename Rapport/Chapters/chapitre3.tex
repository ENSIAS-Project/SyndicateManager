\section{Sprint 0: Mise en Oeuvre}
Avant de débuter la conception avec Scrum/agile, il est essentiel d'identifier d'abord les acteurs, de comprendre leurs besoins et attentes, ainsi que de mettre en place la réalisation du backlog.
\subsection{identification des acteurs}
le sprincipaux acteurs sont: 
\begin{itemize}
    \item \textbf{Administrateur (gérant du syndique)} : 
    \begin{enumerate}
        \item gérer la recette;
        \item marquer les dépenses;
        \item marque les cotisations des membres.
    \end{enumerate} 
    \item \textbf{Utilisateur (membre du syndique)} : 
    \begin{enumerate}
        \item consulter la situation.
    \end{enumerate}
\end{itemize}
\subsection{userStories ( Backlog du produit)}
    basant sur les besoins des acteurs on a definis les UserStories suivantes : 
    \begin{itemize}
        \item En tant qu'administrateur, je veux avoir la possibilité de m'authentifier en utilisant mon login et mon mot de passe de manière sécurisée afin d'exploiter l'application.
        \item En tant qu'utilisateur de l'application, je veux avoir la possibilité de m'authentifier en utilisant mon login et mon mot de passe de manière sécurisée afin d'exploiter l'application.    
        \item En tant qu'utilisateur, je veux avoir la possibilité de créer un nouveau compte utilisateur afin de l'utiliser.
        \item En tant qu'utilisateur, je veux avoir la possibilité de réinitialiser mon mot de passe afin de retrouver l'accès a mon compte.
        \item En tant qu'administrateur, je veux avoir la possibilité de désactiver le compte d'un utilisateur de l'application afin d'annuler les droits d'accès a un utilisateur.
        \item En tant qu'administrateur, je veux avoir la possibilité d'ajouter un type de dépenses à la liste des dépenses prédéfinies dans l'application afin de mieux catégoriser les dépenses.
        \item En tant qu'administrateur, je veux avoir la possibilité de modifier un type de dépenses à la liste des dépenses prédéfinies dans l'application afin de mieux catégoriser les dépenses.
        \item En tant qu'administrateur, je veux avoir la possibilité d'ajouter une contribution à la liste des contributions en spécifiant l'utilisateur, le montant, et la date (qui doit être le jour même par défaut) afin d'enregistrer les contributions.
        \item En tant qu'administrateur, je veux avoir la possibilité de supprimer une contribution de la liste des contributions afin de rectifier les enregistrements incorrects.
        \item En tant qu'administrateur, je veux avoir la possibilité d'ajouter une dépense à la liste des dépenses en spécifiant le type, le montant, et la date (qui doit être le jour même par défaut) afin d'enregistrer les dépenses.
        \item En tant qu'administrateur, je veux avoir la possibilité de supprimer une dépense de la liste des dépenses afin de rectifier les enregistrements incorrects.
        \item En tant qu'utilisateur, je veux voir les revenus, les dépenses, et le solde de chaque mois afin de mieux saisir la situation.
        \item En tant qu'utilisateur, je veux voir en détail les dépenses et les revenus par mois afin de mieux saisir la situation de chaque mois.
    \end{itemize} 
\subsection{prototypage des interfaces}
\subsubsection{l'authentification}
\begin{figure}[!htbp]
\begin{minipage}[t]{0.25\textwidth}    %% b or t, default is c
        \includegraphics[width=\linewidth]{prototype/signup screen.jpeg}
        \caption{ prototype de l'ecran d'inscription}
  \end{minipage}%
  \begin{minipage}{0.10\textwidth}
    \hfill
  \end{minipage}
\begin{minipage}[t]{0.25\textwidth}
        \includegraphics[width=\linewidth]{prototype/sign in.jpeg}
        \caption{ prototype de l'ecran de connexion}
\end{minipage}%
\begin{minipage}{0.10\textwidth}
    \hfill
\end{minipage}
  \begin{minipage}[t]{0.25\textwidth}
        \includegraphics[width=\linewidth]{prototype/reset password.jpeg}
        \caption{ prototype de l'ecran de réinitialisation de mot de passe}
\end{minipage}
\end{figure}
\newpage
\subsubsection{ajout de cotisation et de depenses}
\begin{figure}[h]
    \begin{minipage}[t]{0.25\textwidth}    %% b or t, default is c
            \includegraphics[width=\linewidth]{prototype/side nav.jpeg}
            \caption{ prototype de menu latérale}
      \end{minipage}%
      \begin{minipage}{0.10\textwidth}
        \hfill
      \end{minipage}
    \begin{minipage}[t]{0.25\textwidth}
            \includegraphics[width=\linewidth]{prototype/add contribution.jpeg}
            \caption{ prototype de l'ecran d'ajout des contributions}
    \end{minipage}%
    \begin{minipage}{0.10\textwidth}
        \hfill
    \end{minipage}
      \begin{minipage}[t]{0.25\textwidth}
            \includegraphics[width=\linewidth]{prototype/add expenses.jpeg}
            \caption{ prototype de l'ecran d'ajout des dépenses}
    \end{minipage}
    \end{figure}
    \newpage
\subsubsection{l'affichage des situations}
\begin{figure}[h!]
    \begin{minipage}[t]{0.25\textwidth}    %% b or t, default is c
            \includegraphics[width=\linewidth]{prototype/admin month view.jpeg}
            \caption{ prototype d'affichage pour l'administrateur}
      \end{minipage}%
      \begin{minipage}{0.10\textwidth}
        \hfill
      \end{minipage}
    \begin{minipage}[t]{0.25\textwidth}
            \includegraphics[width=\linewidth]{prototype/user month view.jpeg}
            \caption{ prototype d'affichage pour l'utilisateur}
    \end{minipage}%
    \begin{minipage}{0.10\textwidth}
        \hfill
    \end{minipage}
      \begin{minipage}[t]{0.25\textwidth}
            \includegraphics[width=\linewidth]{prototype/month details.jpeg}
            \caption{ prototype d'affichage des opérations}
    \end{minipage}
    \end{figure}
\section{Sprint 1}
\subsection{specification fonctionnel}
\subsubsection{S'inscrire}
Cette opération permet à un utilisateur de créer un nouveau compte dans l'application.
l'utilisateur fournit ses informations personnelles telles que son nom,son prènom, son adresse e-mail et un mot de passe pour créer son compte. après une vérification des informations fournies le nouvel utilisateur sera enregister dans la  base de données.
\subsubsection{Se connecter}
L'opération de connexion permet à un utilisateur enregistré d'accéder à son compte. L'utilisateur saisit son adresse e-mail et son mot de passe dans les champs prévus à cet effet. L'application vérifie les informations saisies et authentifie l'utilisateur si les informations sont correctes
\subsubsection{Réinitialisation le mot de passe}
Cette opération permet à un utilisateur de réinitialiser son mot de passe en cas d'oubli. L'utilisateur fournit son adresse e-mail associée à son compte. L'application envoie un lien de réinitialisation par e-mail à l'utilisateur. L'utilisateur peut ensuite cliquer sur le lien pour choisir un nouveau mot de passe et le mettre à jour dans la base de données.
\subsection{Sprint Backlog}
\begin{center}
    \begin{tabular}{ | m{1cm} | m{9cm}| m{2cm} | m{2cm} |} 
     \hline
     GitHub ID & Sprint Backlog & Acteur & Priorité \\ [0.5ex] 
     \hline\hline
     \#2 & en tant qu'administrateur, je veux avoir la possibilité de m'authentifier en utilisant mon login et mon mot de passe de manière sécurisée afin d'exploiter l'application & Administrateur & ELVEE \\ 
     \hline
     \#3  & En tant qu'utilisateur de l'application, je veux avoir la possibilité de m'authentifier en utilisant mon login et mon mot de passe de manière sécurisée afin d'exploiter l'application & Utilisateur & ELEVEE \\
     \hline
     \#4 & En tant qu'utilisateur, je veux avoir la possibilité de créer un nouveau compte utilisateur afin de l'utiliser & Utilisateur & MOYEN \\
     \hline
     \#5 & En tant qu'utilisateur, je veux avoir la possibilité de réinitialiser mon mot de passe afin de retrouver l'accès a mon compte & Utilisateur & MOYEN \\ [1ex] 
     \hline
    \end{tabular}
    \end{center}
\subsection{conception}
\subsubsection{diagramme de cas d'utilisation}
\begin{figure}[h!]
    \begin{center}
    \begin{tikzpicture}
        \begin{umlsystem}[x=3, fill=red!10]{Sprint1}
            \umlusecase[name = CONNECT]{se connecter}
            \umlusecase[name = SIGNUP, y = -2]{s'inscrire}
            \umlusecase[name = renewPassword, y = -4]{renouvler le mot de passe} 
            \umlusecase[name = AUTH,x = 5,y = -1]{s'authentifier}  
            \end{umlsystem}
        
            \umlactor[x=-2]{utilisateur}
            \umlactor[x =-2,y=-3]{administrateur}
        
        \umlassoc{utilisateur}{CONNECT}
        \umlassoc{utilisateur}{SIGNUP}
        \umlassoc{utilisateur}{renewPassword}
        \umlassoc{administrateur}{CONNECT}
        \umlassoc{administrateur}{renewPassword}

        \umlHVinclude[name=incl]{CONNECT}{AUTH}
       % \umlHVinclude[name=incl]{addSpending}{auth}
        %\umlHVinclude[name=incl,anchor1=30 , anchor2=300]{addBudget}{auth}

    \end{tikzpicture}
    \caption{le diagramme de cas d'utilisation pour Sprint 1}
\end{center}
    \label{fig : Sprint 1 usecase }
\end{figure}

%and this is a reference to the usecase fig : \ref{fig : usecase 1}
\subsubsection{diagramme de classe}
\begin{figure}[h!]
        \begin{tikzpicture}
        
        \umlclass[x=0,y=0,type=interface,scale = 0.55,fill=blue!10]{AccountService}{
          }{ 
            + authenticate (login : LoginUiModel, onResult: (User) -> Unit) :Unit\\
            + logout() : Unit \\
            + Register(Register: RegisterUiModel, onResult: (User) -> Unit) : Unit \\
            + reset(email: String,onResult: () -> Unit) : Unit \\
          }
          \umlclass[x=9,y=0,scale = 0.55,fill=blue!10]{FireBaseAccountService}{
          }{ 
            - resetPasswordListener(task: Task<Void>, onResult: () -> Unit) :Unit\\
            - loginListerner(task: Task<AuthResult>,onResult: (User) -> Unit) : Unit \\
            - registerListerner(t:Task<AuthResult>,r:RegisterUiModel,onResult:(User)->Unit) :Unit \\
            - setUserData(t:Task<AuthResult>,u:String?,r:RegisterUiModel,onResult:(User)->Unit) : Unit \\
            - authException(e:Exception) :AuthException\\
            - getUserData(task:Task<AuthResult>,uid:String?,onResult:(User) -> Unit) :Unit\\
            - ongetUserDataSucessListener(document: DocumentSnapshot,onResult: (User) -> Unit) :Unit\\
            - onFirestoreException(e: java.lang.Exception) :Unit\\
          }
          \umlclass[x=1,y=-3,scale = 0.55,,fill=green!10]{RegisterUiModel}{
            + prenom : String \\
            + nom : String \\
            + email : String \\
            + password : String
          }{}
            \umlclass[x=-2,y=-3,scale = 0.55,fill=green!10]{LoginUiModel}{
            + email : String \\
            + password : String
          }{}
              \umlclass[x=1,y=-6,scale = 0.55,fill=orange!10]{LoginUiState}{
            + email : String \\
            + password : String \\
            + logging : Boolean \\
            + validMail : Boolean
          }{}
              \umlclass[x=4,y=-6,scale = 0.55,fill=orange!10]{ResetUiState}{
            + email : String \\
            + isMailValid : Boolean
          }{}
          
              \umlclass[x=-2,y=-6,scale = 0.55,fill=orange!10]{RegisterUiState}{
            + prenom : String \\
            + nom : String \\
            + email : String \\
            + password : String \\
            + validMail : Boolean
          }{}
        
          \umlclass[x=8,y=-3,scale = 0.55,fill=red!10]{Exception}{
          }{
          } 
          
          \umlclass[x=8,y=-5,type=abstract,scale = 0.55,fill=red!10]{AuthException}{
          }{
            + getmessage() : Int
          }
            \umlclass[x=12,y=-3,scale = 0.55,fill=red!10]{DeadLineExceeded}{
          }{
          }
             \umlclass[x=12,y=-4,scale = 0.55,fill=red!10]{InvalidCredentialsException}{
          }{
          }
        
           \umlclass[x=12,y=-5,scale = 0.55,fill=red!10]{InvalidUserIdException}{
          }{
          }
        
          \umlclass[x=12,y=-6,scale = 0.55,fill=red!10]{MalFormatedEmailException}{
          }{
          }
        
            \umlclass[x=12,y=-7,scale = 0.55,fill=red!10]{RegisterPasswordMismatchException}{
          }{
          }
        
              \umlclass[x=12,y=-8,scale = 0.55,fill=red!10]{UndefinedException}{
          }{
          } 
        
          \umlclass[x=12,y=-9,scale = 0.55,fill=red!10]{UserDataMissingException}{
          }{
          } 
        
            \umlclass[x=4,y=-3,scale = 0.55,fill=white!10]{User}{
            - IS\_ADMIN:Boolean \\
            name : String \\
            familyname : String \\
            id : String \\
            email : String
          }{
          } 
        \umlclass[x=8,y=-7,scale = 0.55,fill=yellow!10]{ViewModel}{
          }{
          } 
        
        \umlclass[x=4,y=-11,scale = 0.55,fill=yellow!10]{AuthViewModel}{
            + loginUistate :  MutableState<LoginUiState> \\
            + registerUistate : MutableState<RegisterUiState> \\
            + resetUiState : MutableState<ResetUiState> \\
          }{
            +Constructor(accountService : AccountService) \\ % TODO: find the correct way to represent the constructor
            + login(openAndPopUp:(String,String)->Unit,toggleAdminUservalues:(isadmin:Boolean,logged:Boolean)->Unit)  : Unit \\
            + loginExceptionHandler(e: AuthException) : Unit \\
            + setUser(user: User) : Unit \\
            + signupscreen(open: ( String) -> Unit) : Unit \\
            + setLoginEmail(newemail:String) : Unit \\
            + setLoginPassword(newpass:String) : Unit \\
            + onLoginEmailValidation(valid: Boolean) :  Unit \\
            + register(openAndPopUp: (String, String) -> Unit) : Unit \\
            + setRegisterName(s: String) : Unit \\
            + setFamilyname(s: String) : Unit \\
            + setEmail(s: String) : Unit \\
            + setRegisterPass(s: String) : Unit \\
            + setVerificationPass(s: String) : Unit \\
            + onRegisterEmailValidation(valid: Boolean) : Unit \\
            + resetSetEmail(s: String) : Unit \\
            + resetPassword(openAndPopUp: (String, String) -> Unit) : Unit \\
            + onResetEmailValidation(b: Boolean) : Unit \\
            + resetPasswordScreen(open: (String) -> Unit) : Unit 
          } 
        
        \umlinherit{AuthViewModel}{ViewModel}
        \umlinherit{FireBaseAccountService}{AccountService}
        \umlinherit{AuthException}{Exception}
        \umlinherit{DeadLineExceeded}{AuthException}
        \umlinherit{InvalidCredentialsException}{AuthException}
        \umlinherit{InvalidUserIdException}{AuthException}
        \umlinherit{MalFormatedEmailException}{AuthException}
        \umlinherit{RegisterPasswordMismatchException}{AuthException}
        \umlinherit{UndefinedException}{AuthException}
        \umlinherit{UserDataMissingException}{AuthException}
        
        \umlaggreg[]{AuthViewModel}{LoginUiState}
        \umlaggreg[]{AuthViewModel}{RegisterUiState}
        \umlaggreg[]{AuthViewModel}{ResetUiState}
        
        \umlassoc[]{User}{AccountService}
        \umlassoc[]{AuthException}{AuthViewModel}
        \umlassoc[]{User}{AuthViewModel}
        \umlassoc[]{LoginUiModel}{AuthViewModel}
        \umlassoc[]{RegisterUiModel}{AuthViewModel}
        \umlassoc [geometry=|-, anchors=-160 and 170,] {AccountService}{AuthViewModel}
        
        %\umlunicompo[geometry=-|, arg=titi, mult=*, pos=1.7, stereo=vector]{D}{C}
        %\umlimport[geometry=|-, anchors=90 and 50, name=import]{sp2}{sp1}
        %\umlinherit[geometry=-|]{D}{B}
        \end{tikzpicture}
        \caption{le diagramme de Class pour Sprint 1}
        \label{fig : Class diagram Sprint 1}
    \end{figure}
\subsection{Realisation}
\subsubsection{interface de connexion}
\subsubsection{interface d'inscription}
\subsubsection{interface de renitialisation de mot de passe}



\section{Sprint 2}
\subsection{specification fonctionnel}
\subsection{Sprint Backlog}
\subsection{conception}
\subsubsection{diagramme de cas d'utilisation}
\begin{figure}[h]
    \centering
    \begin{tikzpicture}
        \begin{umlsystem}[x=3, fill=orange!10]{Sprint 2}
            \umlusecase[name = MONTH]{voir la situation des mois}
            \umlusecase[name = OPERATION, y = -2]{voir les opérations}
            \end{umlsystem}
            \umlactor[x=-3,y=1]{utilisateur}
            \umlactor[x =-3,y=-2]{administrateur}

            \umlinherit[]{administrateur}{utilisateur}
        \umlassoc{utilisateur}{MONTH}
        \umlassoc{utilisateur}{OPERATION}

       % \umlHVinclude[name=incl]{addSpending}{auth}
        %\umlHVinclude[name=incl,anchor1=30 , anchor2=300]{addBudget}{auth}

    \end{tikzpicture}
    \caption{le diagramme de cas d'utilisation pour Sprint 2}
    \label{fig : Sprint 2 usecase }
\end{figure}

%and this is a reference to the usecase fig : \ref{fig : usecase 1}
\subsubsection{diagramme de classe}
\begin{figure}[h]
        \begin{tikzpicture}

        \umlclass[x=-1,y=3,type=interface,scale = 0.8,fill=blue!10]{DataService}{
            + users : Flow<List<User>$ $>\\
            + expensesTypes: Flow<List<SpendType>$ $>\\
            + monthList: Flow<List<Month>$ $>\\
          }{ 
            + getOperationsFlow(id: String): Flow<List<Operation>$ $> :Unit\\ 
          }

          \umlclass[x=0,y=-2,scale=0.8,fill=blue!10]{FireBaseDataService}{
            + auth: FirebaseAuth\\
            + store : FirebaseFirestore\\
            - MONTH\_DATA\_COLLECTION : String\\
            - SPEND\_TYPES\_COLLECTION : String\\
            - LIST : String\\
          }{
            + Constructor(auth: FirebaseAuth,store : FirebaseFirestore)\\
            - getexpenseType(id:String):SpendType\\
            - getUser(id:String):User\\
            - onFirestoreException(e: java.lang.Exception) : Unit
          } 
          \umlclass[x=7,y=3,scale = 0.8,fill=white!10]{Month}{
            + id :String \\
            - prevBalance : Long \\
            + currBalance: Long \\
            + monthDate : Date \\
            + debit : Long \\
            + credit : Long
          }{
          } 

          \umlclass[x=9,y=-1,scale = 0.8,fill=white!10]{SpendType}{
            + id :String \\
            - prevBalance : Long \\
            + currBalance: Long \\
            + monthDate : Date \\
            + debit : Long \\
            + credit : Long
          }{
          } 
          \umlclass[x=8,y=-5,scale = 0.8,fill=white!10]{User}{
            - IS\_ADMIN:Boolean \\
            name : String \\
            familyname : String \\
            id : String \\
            email : String
          }{
          } 

          \umlclass[x=9,y=-9,scale = 0.8,fill=white!10]{Operation}{
            + id :String \\
            - ref:String \\
            + type :String \\
            + value : Long \\
            + date : Date  \\
            + Spendtype : SpendType \\
            + user : User \\
          }{
          } 
          \umlclass[x=0,y=-5,scale = 0.8,fill=yellow!10]{ViewModel}{
            }{
            } 
          
          \umlclass[x=1,y=-9,scale = 0.8,fill=yellow!10]{MonthViewModel}{
              - dataService: DataService\\
              + monthList :  Flow<List<Month>$ $> \\
            }{
              + constructor(dataService: DataService) \\ % TODO: find the correct way to represent the constructor
              + onMonthSelect(mId: String,m:Int,y:Int, open: (String) -> Unit): Unit \\
              + getOperationFlow(id:String?): Flow<List<Operation>$ $> \\
            } 
        
        \umlinherit[geometry=|-|,anchors=50 and -50]{MonthViewModel}{ViewModel}
        \umlinherit[geometry=|-|]{FireBaseDataService}{DataService}
        
        \umlaggreg[geometry=|-,anchors=80 and 0]{Operation}{User}
        \umlaggreg[geometry=|-,anchors=50 and 0]{Operation}{SpendType}
        \umlaggreg[geometry=-|,anchors=180 and -170]{MonthViewModel}{DataService}
        \umlaggreg[]{MonthViewModel}{Operation}
        \umlaggreg[geometry=|-,anchors=20 and 0]{MonthViewModel}{Month}
        \end{tikzpicture}
        \caption{le diagramme de Class pour Sprint 2}
        \label{fig : Class diagram Sprint 1}
    \end{figure}
\subsection{Realisation}
\subsubsection{interface de situation des mois}
\subsubsection{interface des opérations}



\section{Sprint 3}
\subsection{specification fonctionnel}
\subsection{Sprint Backlog}
\subsection{conception}
\subsubsection{diagramme de cas d'utilisation}
\begin{figure}[h]
    \centering
    \begin{tikzpicture}
        \begin{umlsystem}[x=3,fill=gray!10]{Sprint 3}
            \umlusecase[name=depensetype,x=0,y=2,fill=orange!30]{Gérer les types de dépenses }
            \umlusecase[name=depense,x=0,y=-0,fill=red!30]{Gérer les dépenses }
            \umlusecase[name=cotisation,y=-3,fill=green!30]{Gérer les cotisations}
            \umlusecase[name=addexpense,x=7,y=-1,fill=red!10]{Ajouter dépense}
            \umlusecase[name=updatexpense,x=7,y=-2,fill=red!10]{Supprimer dépense}

            \umlusecase[name=addexpensetype,x=7,y=1,fill=orange!10]{Ajouter type de dépense}  
            \umlusecase[name=modifyexpensetype,x=7,y=3,fill=orange!10]{Modifier type de dépense}  

            \umlusecase[name=addcontrib,x=7,y=-4,fill=green!10]{Ajouter cotisation}
            \umlusecase[name=deletecontrib,x=7,y=-5,fill=green!10]{Supprimer cotisation}
            \end{umlsystem}
            \umlactor[x=-2,y=-1]{administrateur}

        \umlassoc[anchor1=0 , anchor2=180]{administrateur}{depensetype}
        \umlassoc[anchor1=0 , anchor2=180]{administrateur}{depense}
        \umlassoc[anchor1=0 , anchor2=180]{administrateur}{cotisation}
        \umlHVextend[name=ext,anchor1=180 , anchor2=-90]{addexpensetype}{depensetype}
        \umlHVextend[name=ext,anchor1=180 , anchor2=90]{modifyexpensetype}{depensetype}
        \umlHVextend[name=ext,anchor1=180 , anchor2=-10]{addexpense}{depense}
        \umlHVextend[name=ext,anchor1=180 , anchor2=-90]{updatexpense}{depense}
        \umlHVextend[name=ext,anchor1=180 , anchor2=-10]{addcontrib}{cotisation}
        \umlHVextend[name=ext,anchor1=180 , anchor2=-30]{deletecontrib}{cotisation}

       % \umlHVinclude[name=incl]{addSpending}{auth}
        %\umlHVinclude[name=incl,anchor1=30 , anchor2=300]{addBudget}{auth}

    \end{tikzpicture}
 \caption{Le diagramme de Class pour Sprint 3}
\end{figure}

%and this is a reference to the usecase fig : \ref{fig : usecase 1}

\subsubsection{diagramme de classe}
\begin{figure}[h!]
        \begin{tikzpicture}

        \umlclass[x=0,y=-3,type=interface,scale = 0.8,fill=blue!10]{DataService}{
            + users : Flow<List<User>$ $>\\
            + expensesTypes: Flow<List<SpendType>$ $>\\
            + monthList: Flow<List<Month>$ $>\\
          }{ 
            + getOperationsFlow(id: String): Flow<List<Operation>$ $> :Unit\\ 
            + addExpenseType(name: String, onResult: () -> Unit) : Unit\\
            + updateExpenseType(id: String, newname: String, onResult: () -> Unit) : Unit \\
            + addOperation (op :Operation, onResult: () -> Unit) : Unit \\
            + removeOperation(op: Operation, onResult: ()->Unit) : Unit 
          }

          \umlclass[x=0,y=3,scale=0.8,fill=blue!10]{FireBaseDataService}{
            + auth: FirebaseAuth\\
            + store : FirebaseFirestore\\
           % - OPVALUE : String \\
           % - OPTYPE : String \\
           % - OPDATE : String \\
           % - OPREF : String \\
           % - PREV\_BALANCE : String \\
           % - CURR\_BALANCE : String \\
           % - DEBIT : String \\
           % - CREDIT : String \\
           % - MONTHDATE : String \\
           % - LIST : String \\
           % - USERS : String \\
           % - USERS : String \\
            - MONTH\_DATA\_COLLECTION : String\\
            - SPEND\_TYPES\_COLLECTION : String\\
            - LIST : String\\
          }{
            + Constructor(auth: FirebaseAuth,store : FirebaseFirestore)\\
            - getexpenseType(id:String) : SpendType\\
            - getUser(id:String) : User\\
            - onFirestoreException(e: java.lang.Exception) : Unit\\
            - addExpenseType(name: String, onResult: () -> Unit) : Unit\\
            - updateExpenseType(id: String, newname: String, onResult: () -> Unit) : Unit\\
            - updateMonth(m: Month, onResult: () -> Unit) : Unit\\
            - addMonth(m: Month) : Month\\
            - getMonthDateBasedOnOpDate(date:Date) : Date\\
            - getMonthByDateOrCreateNewOne(time: Date): Month  
          } 
          \umlclass[x=8,y=3,scale = 0.6,fill=white!10]{Month}{
            + id :String \\
            + prevBalance : Long \\
            + currBalance: Long \\
            + monthDate : Date \\
            + debit : Long \\
            + credit : Long
          }{
          }

          \umlclass[x=9,y=0,scale = 0.6,fill=white!10]{SpendType}{
            + id : String\\
            + name : String
          }{
          } 
          \umlclass[x=8,y=-3,scale = 0.6,fill=white!10]{User}{
            - IS\_ADMIN:Boolean \\
            name : String \\
            familyname : String \\
            id : String \\
            email : String
          }{
          } 

          \umlclass[x=1,y=-7,scale = 0.6,fill=orange!10]{ContributionUiState}{
            + user : User \\
            + date : Date \\
            + amount : Int \\
            + pendingOperation : Boolean \\
          }{}

          \umlclass[x=5,y=-7,scale = 0.6,fill=orange!10]{ExpenseuiState}{
            + type : String \\
            + date : Date \\
            + amount : Int \\
            + visibleName : String \\
            + ref : String \\
            + pendingOperation : Boolean \\
          }{}

          \umlclass[x=9,y=-12,scale = 0.6,fill=white!10]{Operation}{
            + id :String \\
            - ref:String \\
            + type :String \\
            + value : Long \\
            + date : Date  \\
            + Spendtype : SpendType \\
            + user : User \\
          }{
          } 
          \umlclass[x=-3,y=-7,scale = 0.6,fill=yellow!10]{ViewModel}{
            }{
            } 
          
          \umlclass[x=-1,y=-12,scale = 0.6,fill=yellow!10]{OperationViewModel}{
              - dataService: DataService\\
              - CONTRIBUTION : String \\
              - EXPENSE : String \\
              - users : Flow<List<User>$ $> \\
              - contribiutionUiState :  MutableState<ContributionUiState> \\
              - expenseUiState : MutableState<ExpenseuiState> \\
              - expensesTypes : Flow<List<SpendType>$ $> \\
              + monthList :  Flow<List<Month>$ $> \\
            }{
              + constructor(dataService: DataService) \\ % TODO: find the correct way to represent the constructor
              + addExpense() : Unit \\
              + addexpenseResult() : Unit \\
              + addcontributionResult() : Unit\\
              + addExpenseType(id: String) : Unit \\
              + modifyExpenseType(id: String, name: String) : Unit\\
              + setNewVal(newVal: String) : Unit\\
              + onContribValueChange(newval: String) : Unit \\
              + addContribution() : Unit \\
            } 
        
        \umlinherit[geometry=|-|,anchors=50 and -50]{OperationViewModel}{ViewModel}
        \umlinherit[geometry=|-|]{FireBaseDataService}{DataService}
        
        \umlaggreg[geometry=|-,anchors=80 and 0]{Operation}{User}
        \umlaggreg[geometry=|-,anchors=50 and 0]{Operation}{SpendType}
        \umlaggreg[geometry=-|,anchors=30 and -60]{OperationViewModel}{SpendType}
        \umlaggreg[draw=black!30]{OperationViewModel}{Month}
        \umlaggreg[geometry=-|,anchors=180 and -170]{OperationViewModel}{DataService}
        \umlassoc[geometry=-|,anchors=20 and 100]{FireBaseDataService}{Month}
        \umlaggreg[geometry=-|,anchors=-20 and 120]{DataService}{Operation}
        \umlaggreg{OperationViewModel}{ContributionUiState}
        \umlaggreg{OperationViewModel}{ExpenseuiState}
        \umlassoc[geometry=-|,anchors=-20 and 130]{FireBaseDataService}{User}
        %\umlaggreg[geometry=|-,anchors=20 and 0]{MonthViewModel}{Month}
        \end{tikzpicture}
        \caption{Le diagramme de Class pour Sprint 3}
        \label{fig : Class diagram Sprint 3}
    \end{figure}

\subsection{Realisation}
\subsubsection{interface pour gérer les opération}
\paragraph{ajouter type de dépense}
\paragraph{modifier type de dépense}
\paragraph{ajouter de dépense}
\paragraph{ajouter cotisation}
\paragraph{supprimer operation}


