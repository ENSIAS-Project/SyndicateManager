image example:

\begin{figure}[h]
\centering
\includegraphics[scale=0.5]{ensias.png}
\caption{An example of an image}
\label{fig : ensias logo}
\end{figure}

to reference to the previous fugure you can use \ref{fig : ensias logo}

inline figures can be made this way 
\begin{figure}[h]
     \centering
     \begin{subfigure}[b]{0.3\textwidth}
         \centering
         \includegraphics[width=\textwidth]{ensias.png}
         \caption{a caption}
         \label{fig:y equals x}
     \end{subfigure}
     \hfill
     \begin{subfigure}[b]{0.3\textwidth}
         \centering
         \includegraphics[width=\textwidth]{ensias.png}
         \caption{$math cap E=mc^2$}
         \label{fig:three sin x}
     \end{subfigure}
     \hfill
     \begin{subfigure}[b]{0.3\textwidth}
         \centering
         \includegraphics[width=\textwidth]{ensias.png}
         \caption{$y=5/x$}
         \label{fig:five over x}
     \end{subfigure}
        \caption{Three simple graphs}
        \label{fig:three graphs}
\end{figure}

an example of a table :

\begin{table}[h]
\centering
\begin{tabular}{||l | l | l||}
\hline
A & B & C \\
\hline
1 & 2 & 3 \\
4 & 5 & 6 \\
\hline
\hline
\end{tabular}
\caption{very basic table}
\label{tab:abc}
\end{table}
