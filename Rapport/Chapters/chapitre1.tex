\section{Problematique}
La gestion des copropriétés constitue un défi de taille pour de nombreux syndics, requérant une coordination minutieuse d'une pluralité de tâches telles que la communication avec les résidents, la collecte des cotisations, la gestion des dépenses. Ces processus sont fréquemment confrontés à des inefficacités et des erreurs humaines lorsqu'elles sont conduites manuellement ou selon des méthodes conventionnelles.

Dans le cadre de notre cursus académique de fin d'année, nous avons choisi d'investiguer le développement d'une solution innovante pour répondre à ces défis. Ainsi, l'élaboration d'une application mobile spécifiquement  dédiée à la gestion de syndic de copropriété offre une opportunité significative d'optimiser les processus existants, d'améliorer la communication entre les différentes parties prenantes et de renforcer la transparence des opérations.
\section{objectif}
Dans ce contexte, notre projet de fin d'année se fixe pour but  la conception et le développement d'une application mobile de gestion de syndic de copropriété. Notre objectif est de répondre à l'interrogation suivante :
de quelle maniere une application mobile dédiée peut-elle concourir à l'optimisation des procedures de gestion, à l'amélioration de la communication entre les résidents et le syndic, ainsi qu'au renforcement de la transparence des opérations, afin de satisfaire de manière efficiente aux besoins des propriétaires ?

\section{l'analyse des besoins}
Dans cette section, nous procédons à une analyse approfondie des exigences inhérentes à notre application mobile consacrée à la gestion du syndic de copropriété, Cette démarche nous permettra de formuler des recommandations spécifiques pour la conception et le développement de notre solution de gestion de syndic de copropriété, assurant ainsi une réponse adéquate aux défis et aux besoins des utilisateurs. De plus, nous identifierons les fonctionnalités essentielles ainsi que les exigences techniques et de performance.tion{l'analyse des besoins}

\subsection{Besions fonctionnels}
Les besoins fonctionnels définissent les fonctionnalités spécifiques que l'application doit offrir pour répondre aux attentes des utilisateurs et aux exigences opérationnelles. 
Voici une élargissement des exigences fonctionnelles de l'application :
\begin{itemize}
    \item Surveillance de l'état budgétaire: L'application doit permettre aux utilisateurs, en particulier aux administrateurs, de suivre de manière précise et en temps réel l'état financier global de la copropriété. Cela implique de fournir des informations détaillées sur les revenus et les dépenses, ainsi que sur le solde actuel du budget.
    \item Résumé mensuel du budget: Les utilisateurs devraient pouvoir visualiser de manière claire et concise un résumé mensuel du budget, mettant en évidence les entrées et les sorties d'argent, les cotisations perçues, les dépenses engagées et le solde restant.
    \item Historique des transactions mensuelles: Pour une gestion transparente et une traçabilité des finances, il est nécessaire de fournir un historique détaillé des transactions effectuées pour un mois donné. Cela permettra aux utilisateurs de comprendre en détail les mouvements financiers et de vérifier la validité des opérations.
    \item Différenciation des droits d'accè : une distinction claire est établie  entre les utilisateurs réguliers et les administrateurs. Les administrateurs devraient avoir des privilèges étendus pour effectuer des opérations de gestion, tandis que les utilisateurs réguliers devraient avoir un accès restreint aux fonctionnalités essentielles
    \item Fonctionnalités administratives : 
    \begin{itemize}
        \item[+] \textbf{Ajouter des cotisations : }
        Permettre aux administrateurs d'ajouter de nouvelles cotisations et de définir leurs montants respectifs.
        \item[+] \textbf{Modifier des cotisations :}
        Autoriser les administrateurs à apporter des modifications aux cotisations existantes, que ce soit pour ajuster les montants ou les périodicités.
        \item[+] \textbf{Ajouter des dépenses :}
        Offrir aux administrateurs la possibilité d'enregistrer de nouvelles dépenses engagées par la copropriété.
        \item[+] \textbf{Modifier des dépenses :}
        Permettre aux administrateurs de corriger toute erreur ou inexactitude dans les dépenses enregistrées.
        \item[+] \textbf{Ajouter un nouvel utilisateur : }
        Autoriser les nouveaux utilisateurs à créer de nouveaux comptes pour avoir l'accès aux données.
        \item[+] \textbf{Réinitialiser le mot de passe d'un utilisateur :}
        Offrir la possibilité aux utilisateurs de réinitialiser les mots de passes en cas de besoin, garantissant ainsi la sécurité des comptes.
    \end{itemize}
\end{itemize}
\subsection{Besoins non fonctionnels}
concernent les aspects techniques et de performance de l'application notament :
\begin{itemize}
    \item \textbf{Sécurité : } L'application doit garantir la confidentialité et l'intégrité des données.
    \item \textbf{Performance : } L'application doit être rapide et offrir une experience utilisateur optimale.
    \item \textbf{Extensibilité : } Il est important de concevoir l'application de manière à ce qu'elle puisse être étendue avec de nouvelles fonctionnalités à l'avenir.
    \item \textbf{Maintenance : } Il est essentiel de mettre en oeuvre une architecture qui facilite  la maintenance de l'application.
\end{itemize}
