\begin{figure}[h]
    \centering
    \begin{tikzpicture}
        \begin{umlsystem}[x=3, fill=green!10]{gestion des dépenses }
            \umlusecase[name = addSpending]{ajouter dépense}
            \umlusecase[name = addSpendingtype, y = -2]{ajouter type de dépense}
            \umlusecase[name = modifySpending, y = -4]{rectifier dépense} 
            \umlusecase[name = deleteSpending, y = -6]{supprimer dépense}
            \umlusecase[name = auth,x = 5,y = -1]{s'authentifier}  
            \end{umlsystem}
        
            \umlactor[x =-2,y=-3]{administrateur}
        
        \umlassoc{administrateur}{addSpending}
        \umlassoc{administrateur}{addSpendingtype}
        \umlassoc{administrateur}{modifySpending}
        \umlassoc{administrateur}{deleteSpending}
        \umlHVinclude[name=incl,anchor1=0 , anchor2=40]{addSpending}{auth}
        \umlHVinclude[name=incl]{addSpendingtype}{auth}
        \umlHVinclude[name=incl,anchor1=0 , anchor2=350]{modifySpending}{auth}
        \umlHVinclude[name=incl,anchor1=0 , anchor2=320]{deleteSpending}{auth}
        %\umlHVinclude[name=incl,anchor1=30 , anchor2=300]{addBudget}{auth}
 
    \end{tikzpicture}
    \caption{le diagramme de cas d'utilisation detaillé pour les dépenses}
    \label{fig : usecase 2}
\end{figure}

%and this is a reference to the usecase fig : \ref{fig : usecase 2}