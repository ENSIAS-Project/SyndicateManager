\begin{figure}[h!]
    \begin{center}
    \begin{tikzpicture}
        \begin{umlsystem}[x=3, fill=red!10]{Sprint1}
            \umlusecase[name = CONNECT]{se connecter}
            \umlusecase[name = SIGNUP, y = -2]{s'inscrire}
            \umlusecase[name = renewPassword, y = -4]{réinitialiser le mot de passe} 
            \umlusecase[name = AUTH,x = 5,y = -1]{s'authentifier}  
            \end{umlsystem}
        
            \umlactor[x=-2]{utilisateur}
            \umlactor[x =-2,y=-3]{administrateur}
        
        \umlassoc{utilisateur}{CONNECT}
        \umlassoc{utilisateur}{SIGNUP}
        \umlassoc{utilisateur}{renewPassword}
        \umlassoc{administrateur}{CONNECT}
        \umlassoc{administrateur}{renewPassword}

        \umlHVinclude[name=incl]{CONNECT}{AUTH}
       % \umlHVinclude[name=incl]{addSpending}{auth}
        %\umlHVinclude[name=incl,anchor1=30 , anchor2=300]{addBudget}{auth}

    \end{tikzpicture}
    \caption{Le diagramme de cas d'utilisation pour Sprint 1}
\end{center}
    \label{fig : Sprint 1 usecase }
\end{figure}

%and this is a reference to the usecase fig : \ref{fig : usecase 1}
